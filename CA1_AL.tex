\documentclass{article}
\usepackage{blindtext}
\usepackage{listings}
\usepackage{graphicx}
\usepackage{amsmath}	
\usepackage{xcolor}

\title{Computing Assignment}
\author{Hyun Soo Lee}
\begin{document}
\maketitle

\lstset{language=Matlab}
The following are the equations of interest:
$$u_t + u\cdot\nabla u = -\nabla p$$
$$\nabla \cdot u = 0$$
$$u \cdot n = 0$$
\section*{Task1}
On the domain $\Omega_1 = [0, 1] \times [0, 1]$, we examine the solutions:
$$u_1(x, y) = 1 - 2\cos(2\pi x)\sin(2\pi y)$$
$$u_2(x, y) = 1 + 2\sin(2\pi y)\cos(2\pi y)$$
$$p_1(x, y) = -\cos(4\pi x) - \cos(4\pi y)$$
The m-file Problem1.m solves the Poisson problem with initial condition
$$\vec{U}_0 = (u^e_{i, j}, v^e_{i, j}) + (u'_{i, j}, v'_{i, j})$$
where $(u'_{i, j}, v'_{i, j})$ is a random perturbation. Once I obtain the solution of the Poisson
problem, $\phi$, I update my velocity field.
$$\vec{U}^* = \vec{U}_0 - \nabla \phi$$
Finally, I compute my error by comparing the updated velocity with the velocity field of the 
steady solution.

\section*{Task 2}
For the the following task, I need to show
$$u_1(x, y) = 1 - 2\cos(2\pi x)\sin(2\pi y)$$
$$u_2(x, y) = 1 + 2\sin(2\pi y)\cos(2\pi y)$$
$$p_1(x, y) = -\cos(4\pi x) - \cos(4\pi y)$$
are the solution of incompressible Euler Equation.\newline
Using Einstein notation we see that 
$$\vec{u}\cdot\nabla\vec{u} = u_i\frac{\partial u_j}{\partial x_i}$$

Thus, componentwise,
$$\vec{u}\cdot\nabla\vec{u} = (u\frac{\partial u}{\partial x} + v\frac{\partial u}{\partial y}, u\frac{\partial v}{\partial x} + v\frac{\partial v}{\partial y})^T$$
Instead of working with the steady equations, work with the actual solutions and plug in $t = 0$ at the end of the calculation.
$$u(x, y, t) = 1 - 2\cos(2\pi(x-t))\sin(2\pi(y-t))$$
$$v(x, y, t) = 1 + 2\sin(2\pi(x-t))\cos(2\pi(y-t))$$
$$p(x, y, t) = -\cos(4\pi(x-t)) - \cos(4\pi(y-t))$$

Note that
$$u_t = -4\pi\sin(2\pi(x-t))\sin(2\pi(y-t)) + 4\pi\cos(2\pi(x-t))\cos(2\pi(y-t))$$
$$u_x = 4\pi\sin(2\pi(x-t))\sin(2\pi(y-t))$$
$$u_y = -4\pi\cos(2\pi(x-t))\cos(2\pi(y-t))$$

Similarly,
$$v_t = -4\pi\cos(2\pi(x-t))\cos(2\pi(y-t)) + 4\pi\sin(2\pi(x-t))\sin(2\pi(y-t))$$
$$v_x = 4\pi\cos(2\pi(x-t))\cos(2\pi(y-t))$$
$$v_y = -4\pi\sin(2\pi(x-t))\sin(2\pi(y-t))$$
Then, 
$$u_t + u\frac{\partial u}{\partial x} + v\frac{\partial u}{\partial y} = -8\pi\sin(2pi(x-t))\cos(2pi(x - t))$$

$$v_t + u\frac{\partial v}{\partial x} + v\frac{\partial v}{\partial y} = -8\pi\sin(2pi(y-t))\cos(2pi(y - t))$$
Finally, 
$$\nabla p(x, y, t) = (\frac{\partial p}{\partial x}, \frac{\partial p}{\partial y})^T$$

$$\frac{\partial p}{\partial x} = 4\pi\sin(4\pi(x-t))$$
$$\frac{\partial p}{\partial y} = 4\pi\sin(4\pi(y-t))$$
By applying $\sin(2\theta) = 2sin(\theta)cos(\theta)$, 
$$-8\pi\sin(2pi(x-t))\cos(2pi(x - t)) = -4\pi\sin(4\pi(x-t))$$
$$-8\pi\sin(2pi(y-t))\cos(2pi(y - t)) = -4\pi\sin(4\pi(y-t))$$
Therefore, the Euler equation is satisfied. By plugging in t = 0,
$$u_1(x, y) = 1 - 2\cos(2\pi x)\sin(2\pi y)$$
$$u_2(x, y) = 1 + 2\sin(2\pi y)\cos(2\pi y)$$
$$p_1(x, y) = -\cos(4\pi x) - \cos(4\pi y)$$
satisfies the Euler Equation. 
Now, incompressiblity of the steady solution will be shown. 
$$\frac{\partial u_1}{\partial x} = 4\pi\sin(2\pi x)\sin(2\pi y)$$
$$\frac{\partial u_2}{\partial y} = -4\pi\sin(2\pi x)\sin(2\pi y)$$
Thus, $$\frac{\partial u_1}{\partial x} + \frac{\partial u_2}{\partial y} = 0$$


\section*{Task 3}
We need to check that $\Gamma$ equals the integral of the vorticity $w(x, y)$ given on page 174.
$$w(x, y) = 2\pi(e^{-2(x^2 + (y-1)^2)} + e^{-2(x^2 + (y+1)^2)} )$$
Apply the integral to each component at a time.
$$\int_{-\infty}^{\infty}\int_{-\infty}^{\infty}e^{-2(x^2 + (y-1)^2)}dydx = \int_{-\infty}^{\infty} e^{-2(y-1)^2}dy\int_{-\infty}^{\infty} e^{-2x^2}dx$$
Each term can be computed by applying a u-sub.
$$\int_{-\infty}^{\infty} e^{-2x^2}dx = \frac{1}{\sqrt{2}} \int_{-\infty}^{\infty}e^{-u^2}du = \sqrt{\frac{\pi}{2}}$$
$$\int_{-\infty}^{\infty} e^{-2(y-1)^2}dy = \frac{1}{\sqrt{2}} \int_{-\infty}^{\infty}e^{-u^2}du = \sqrt{\frac{\pi}{2}}$$
Therefore,
$$\int_{-\infty}^{\infty}\int_{-\infty}^{\infty}e^{-2(x^2 + (y-1)^2)}dydx = \frac{\pi}{2}$$
Applying the same approach,
$$\int_{-\infty}^{\infty}\int_{-\infty}^{\infty}e^{-2(x^2 + (y+1)^2)}dydx = \frac{\pi}{2}$$
Therefore,
$$\Gamma = \int_{-\infty}^{\infty}\int_{-\infty}^{\infty} w(x, y)dydx = 2\pi^2$$
\section*{Task 4}
From my experiments, I noticed that my updated $\vec{u}$ loses its divergence free property
as $c \rightarrow 0$ for all grid sizes. 
Once I plugged in c = 0, my code did not work, as the divergence could not be computed.\newline
The following is the result for $h = \frac{1}{10}$.
\begin{center}
    \begin{tabular}{||c c||} 
     \hline
     c value & $L^\infty$ of Divergence  \\ [0.5ex] 
     \hline\hline
     1 & 1.2945e-13 \\ 
     \hline
     0.1 & 2.7711e-13 \\
     \hline
     0.01 & 2.9354e-13 \\
     \hline
     $10^{-3}$ & 1.8741e-13 \\
     \hline
     $10^{-4}$ & 2.3848e-13 \\
     \hline
     $10^{-5}$ &  1.8252e-13 \\
     \hline
     $10^{-6}$ & 3.6504e-13  \\ [1ex] 
     \hline
    \end{tabular}
    \end{center}

\section*{Task 5}
$$u \cdot n = 0, \;\;\; x \in \partial \Omega$$
implies that there is no flow across the boundary.
My updated $u^*$ satisfies
$$u^* \cdot n = 0, \;\;\; x \in \partial \Omega$$
\section*{Task 6}
The following is a broad outline of the various details of the 
numerical method that I have coded. 
\newline
(1) The center of $(i, j)^{th}$ cell is defined by
$$(x_i, y_j) = (h(i-\frac{1}{2}), h(j-\frac{1}{2}))$$
where
$1 \leq i \leq Nx$ and $1 \leq j \leq Nx$. 
\newline
(2) The coordinates of the edges are defined as
$$(x_i, y_j) = (ih, jh)$$
where
$0 \leq i \leq Nx$ and $0 \leq j \leq Nx$. 
\newline
(3) The discrete exact velocities
$$(u_{i, j}^e, v_{i, j}^e)$$
and the discrete initial velocities
$$(u_{i, j}^*, v_{i, j}^*)$$
exist on the cell edges.\newline
On the other hand, the discrete divergence 
$$D(U)_{i, j}$$ exists on the cell centers. 
\newline
Finally, the discrete gradient $G(\phi)_{i, j}$ 
is defined on the cell edges. 
\newline
(4) The definition of the discrete divergence operator in terms
of the discrete velocities is the following. 
$$D(U)_{i,j} = \frac{u_{i,j}^* - u_{i-1,j}^*}{h} + \frac{v_{i,j}^* - v_{i-1,j}^*}{h}$$
On the other hand, the discrete divergence is defined as
$$G(\phi)_{i,j} = (\frac{\phi_{i+1,j} -\phi_{i,j}}{h}, \frac{\phi_{i,j+1} -\phi_{i,j}}{h})$$
\newline
(5) Rate of convergence for my Known Exact Solutions
\begin{center}
    \begin{tabular}{||c c c c c c||} 
     \hline
     N & $h = \frac{1}{N}$ & $L^2$ & $L^2$ growth rate& $L^\infty$&$L^\infty$ growth rate \\ [0.5ex] 
     \hline\hline
     4 & 0.25 & 1.5616 && 0.6036& \\ 
     \hline
     8 & 0.125 & 1.4535 &1.1918& 0.3560& 2.0458\\
     \hline
     16 & 0.0625 & 1.3570 &1.2250& 0.1851& 1.6332\\
     \hline
     32 & 0.0312 & 1.3256 &1.0831& 0.1112& 1.3021\\
     \hline
     64 & 0.0156 & 1.3103 &1.0430& 0.0589& 1.2893\\  
     \hline
     64 & 0.0156 & 1.2968 &1.0398& 0.0282&  1.2601\\ [1ex] 
     \hline
    \end{tabular}
    \end{center}
The table above examines the u-component of the vector field.


\begin{center}
    \begin{tabular}{||c c c c c c||} 
     \hline
     N & $h = \frac{1}{N}$ & $L^2$ & $L^2$ growth rate& $L^\infty$&$L^\infty$ growth rate \\ [0.5ex] 
     \hline\hline
     4 & 0.25 & 1.5452 &&  0.6036& \\ 
     \hline
     8 & 0.125 & 1.4332 &1.2091& 0.3560& 2.0458\\
     \hline
     16 & 0.0625 & 1.3424 &1.2223& 0.1851& 1.6332\\
     \hline
     32 & 0.0312 & 1.3223 &1.0540& 0.1076& 1.3216\\
     \hline
     64 & 0.0156 & 1.3034 &1.0543& 0.0524& 1.3227\\  
     \hline
     64 & 0.0156 & 1.2976 &1.0171& 0.0292&  1.1983\\ [1ex] 
     \hline
    \end{tabular}
    \end{center}
The table above examines the v-component of the vector field.
\end{document}
